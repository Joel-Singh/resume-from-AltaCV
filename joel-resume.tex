%%%%%%%%%%%%%%%%%
% This is an sample CV template created using altacv.cls
% (v1.7.2, 28 August 2024) written by LianTze Lim (liantze@gmail.com). Compiles with pdfLaTeX, XeLaTeX and LuaLaTeX.
%
%% It may be distributed and/or modified under the
%% conditions of the LaTeX Project Public License, either version 1.3
%% of this license or (at your option) any later version.
%% The latest version of this license is in
%%    http://www.latex-project.org/lppl.txt
%% and version 1.3 or later is part of all distributions of LaTeX
%% version 2003/12/01 or later.
%%%%%%%%%%%%%%%%

%% Use the "normalphoto" option if you want a normal photo instead of cropped to a circle
% \documentclass[10pt,a4paper,normalphoto]{altacv}

\documentclass[10pt,a4paper,ragged2e,withhyper]{altacv}
%% AltaCV uses the fontawesome5 and simpleicons packages.
%% See http://texdoc.net/pkg/fontawesome5 and http://texdoc.net/pkg/simpleicons for full list of symbols.

% Change the page layout if you need to
\geometry{left=1.25cm,right=1.25cm,top=1.5cm,bottom=1.5cm,columnsep=1.2cm}

% The paracol package lets you typeset columns of text in parallel
\usepackage{paracol}

% Change the font if you want to, depending on whether
% you're using pdflatex or xelatex/lualatex
% WHEN COMPILING WITH XELATEX PLEASE USE
% xelatex -shell-escape -output-driver="xdvipdfmx -z 0" sample.tex
\iftutex
  % If using xelatex or lualatex:
  \setmainfont{Roboto Slab}
  \setsansfont{Lato}
  \renewcommand{\familydefault}{\sfdefault}
\else
  % If using pdflatex:
  \usepackage[rm]{roboto}
  \usepackage[defaultsans]{lato}
  % \usepackage{sourcesanspro}
  \renewcommand{\familydefault}{\sfdefault}
\fi

% Change the colours if you want to
\definecolor{SlateGrey}{HTML}{2E2E2E}
\definecolor{LightGrey}{HTML}{666666}
\definecolor{DarkPastelRed}{HTML}{450808}
\definecolor{PastelRed}{HTML}{8F0D0D}
\definecolor{GoldenEarth}{HTML}{E7D192}
\colorlet{name}{black}
\colorlet{tagline}{PastelRed}
\colorlet{heading}{DarkPastelRed}
\colorlet{headingrule}{GoldenEarth}
\colorlet{subheading}{PastelRed}
\colorlet{accent}{PastelRed}
\colorlet{emphasis}{SlateGrey}
\colorlet{body}{LightGrey}

% Change some fonts, if necessary
\renewcommand{\namefont}{\Huge\rmfamily\bfseries}
\renewcommand{\personalinfofont}{\footnotesize}
\renewcommand{\cvsectionfont}{\LARGE\rmfamily\bfseries}
\renewcommand{\cvsubsectionfont}{\large\bfseries}


% Change the bullets for itemize and rating marker
% for \cvskill if you want to
\renewcommand{\cvItemMarker}{{\small\textbullet}}
\renewcommand{\cvRatingMarker}{\faCircle}
% ...and the markers for the date/location for \cvevent
% \renewcommand{\cvDateMarker}{\faCalendar*[regular]}
% \renewcommand{\cvLocationMarker}{\faMapMarker*}


% If your CV/résumé is in a language other than English,
% then you probably want to change these so that when you
% copy-paste from the PDF or run pdftotext, the location
% and date marker icons for \cvevent will paste as correct
% translations. For example Spanish:
% \renewcommand{\locationname}{Ubicación}
% \renewcommand{\datename}{Fecha}


%% Use (and optionally edit if necessary) this .tex if you
%% want to use an author-year reference style like APA(6)
%% for your publication list
% \input{pubs-authoryear.cfg}

%% Use (and optionally edit if necessary) this .tex if you
%% want an originally numerical reference style like IEEE
%% for your publication list
\input{pubs-num.cfg}

%% sample.bib contains your publications
\addbibresource{sample.bib}

\begin{document}
\name{Joel Singh}
\tagline{Denison University, QuestBridge Scholar}
%% You can add multiple photos on the left or right
\photoR{2.8cm}{Globe_High}
% \photoL{2.5cm}{Yacht_High,Suitcase_High}

\personalinfo{%
  \email{joelsingh788@gmail.com}
  \phone{352-708-1714}
  \homepage{joel-singh.com}
  % \twitter{@twitterhandle}
  \github{joel-singh}
  %% You can add your own arbitrary detail with
  %% \printinfo{symbol}{detail}[optional hyperlink prefix]
  % \printinfo{\faPaw}{Hey ho!}[https://example.com/]

  %% Or you can declare your own field with
  %% \NewInfoFiled{fieldname}{symbol}[optional hyperlink prefix] and use it:
  % \NewInfoField{gitlab}{\faGitlab}[https://gitlab.com/]
  % \gitlab{your_id}
  %%
  %% For services and platforms like Mastodon where there isn't a
  %% straightforward relation between the user ID/nickname and the hyperlink,
  %% you can use \printinfo directly e.g.
  % \printinfo{\faMastodon}{@username@instace}[https://instance.url/@username]
  %% But if you absolutely want to create new dedicated info fields for
  %% such platforms, then use \NewInfoField* with a star:
  % \NewInfoField*{mastodon}{\faMastodon}
  %% then you can use \mastodon, with TWO arguments where the 2nd argument is
  %% the full hyperlink.
  % \mastodon{@username@instance}{https://instance.url/@username}
}

\makecvheader
%% Depending on your tastes, you may want to make fonts of itemize environments slightly smaller
% \AtBeginEnvironment{itemize}{\small}

%% Set the left/right column width ratio to 6:4.
\columnratio{0.6}

% Start a 2-column paracol. Both the left and right columns will automatically
% break across pages if things get too long.
\begin{paracol}{2}
\cvsection{Experience}

\cvevent{Programming Lead}{East Ridge HS}{Sophmore - Senior}{Clermont, Florida}
\begin{itemize}
\item Made with Java using the declarative WPILIB framework
\item Communicated across subteams with the goal of finishing a competition ready robot
\item Managed the codebase, ensuring code quality, especially for quick changes and testing during comp and regular meets
\item Kept mentors in the loop, eliciting advice and warnings
\end{itemize}

\divider

\cvevent{Student Aid}{East Ridge HS, Admin Offices}{Junior}{Clermont Florida}
\begin{itemize}
\item Did simple administrative work like printing out IDs and answering calls
\item Maintained confidentiality in student information
\item Referred back to a supervisor when necessary
\end{itemize}

\cvsection{Projects}

\cvevent{Project 1}{Funding agency/institution}{}{}
\begin{itemize}
\item Details
\end{itemize}

\divider

\cvevent{Project 2}{Funding agency/institution}{Project duration}{}
A short abstract would also work.

\medskip

% use ONLY \newpage if you want to force a page break for
% ONLY the current column
\newpage


%% Switch to the right column. This will now automatically move to the second
%% page if the content is too long.
\switchcolumn

\cvsection{Awards}

\cvachievement{\faHeartbeat}{100+ hours volunteering}{Mostly for local events and foodbanks}

\divider

\cvachievement{\faBook}{Top 25 class rank out of 600}{Based on GPA}

\divider

\cvachievement{\faBook}{QuestBridge Scholar}{Competitive program, providing a full four-year scholarship}

\cvsection{Strengths}

% Don't overuse these \cvtag boxes — they're just eye-candies and not essential. If something doesn't fit on a single line, it probably works better as part of an itemized list (probably inlined itemized list), or just as a comma-separated list of strengths.

\cvtag{Hard-working}
\cvtag{Eye for detail}\\
\cvtag{Motivator \& Leader}

\divider\smallskip

\cvtag{C++}
\cvtag{Embedded Systems}\\
\cvtag{Statistical Analysis}

\cvsection{Languages}

\cvskill{English}{5}
\divider

\cvskill{Spanish}{4}
\divider

\cvskill{German}{3.5} %% Supports X.5 values.

%% Yeah I didn't spend too much time making all the
%% spacing consistent... sorry. Use \smallskip, \medskip,
%% \bigskip, \vspace etc to make adjustments.
\medskip

\cvsection{Education}

\cvevent{Ph.D.\ in Your Discipline}{Your University}{Sept 2002 -- June 2006}{}
Thesis title: Wonderful Research

\divider

\cvevent{M.Sc.\ in Your Discipline}{Your University}{Sept 2001 -- June 2002}{}

\divider

\cvevent{B.Sc.\ in Your Discipline}{Stanford University}{Sept 1998 -- June 2001}{}

% \divider

\cvsection{Referees}

% \cvref{name}{email}{mailing address}
\cvref{Prof.\ Alpha Beta}{Institute}{a.beta@university.edu}
{Address Line 1\\Address line 2}

\divider

\cvref{Prof.\ Gamma Delta}{Institute}{g.delta@university.edu}
{Address Line 1\\Address line 2}


\end{paracol}


\end{document}
